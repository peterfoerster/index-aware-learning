\documentclass[a4paper, 10pt,
    ]{article}

\usepackage[top=1.75cm, bottom=1.75cm, left=3cm, right=3cm]{geometry}
\usepackage[ngerman]{babel}
\usepackage[T1]{fontenc}

% font
\usepackage[scaled=.98,sups,osf]{XCharter}
\usepackage[scaled=1.04,varqu,varl]{inconsolata}
\usepackage[type1]{cabin}
\usepackage[charter,vvarbb,scaled=1.05]{newtxmath}
\usepackage[cal=boondoxo]{mathalfa}
\linespread{1.04}

\usepackage{tuda-pgfplots}

\pagestyle{empty}
\setlength{\parindent}{0cm}

\newcommand{\mb}[1]{\mathbf{#1}}
\newcommand{\mbt}[1]{\tilde{\mathbf{#1}}}
\newcommand{\mbb}[1]{\bar{\mathbf{#1}}}
\newcommand{\mbh}[1]{\hat{\mathbf{#1}}}
\newcommand{\mr}[1]{\mathrm{#1}}
\newcommand{\T}{{\!\top}}
\newcommand{\ddt}{\frac{\mathrm{d}}{\mathrm{d}t}}
\newcommand{\A}[1]{\mb{A}_\mr{#1}}
\newcommand{\AT}[1]{\mb{A}_\mr{#1}^{\T}}
\newcommand{\qC}{\mb{q}_\mr{C}}
\newcommand{\gR}{\mb{g}_\mr{R}}
\newcommand{\phiL}{\boldsymbol{\phi}_\mr{L}}
\newcommand{\vphi}{\boldsymbol{\varphi}}
\renewcommand{\i}[1]{\mb{i}_\mr{#1}}
\renewcommand{\v}[1]{\mb{v}_\mr{#1}}

\begin{document}
    Dear editors of the International Journal of Circuit Theory and Applications,\\

    thank you for the feedback on our manuscript ``Index-aware learning of circuits". We have reviewed the comments (\textcolor{TUDa-0c}{repeated in gray}) and made changes (\textcolor{red}{highlighted in red}) in order to address the criticism and improve the paper.\\

    \paragraph{Reviewing: 1}
    \begin{itemize}
        \item[1)] \textcolor{TUDa-0c}{Why do you have to linearize $g_R$ in equation (4)? This is not necessary for the well-known index-results for the MNA. Is this linearization important for your approach with the dissection index?}
        \begin{itemize}
            \item We added a footnote in section 2 to emphasize that this linearization is not necessary. It only serves to achieve a clear presentation of the approach by consistently writing MNA in a matrix form.\\
            ``\textcolor{red}{Note that the definition of $\mb{G}$ via the Jacobian only serves to obtain a matrix form in \dots\ and has no impact on the general approach. As such, one may also work directly with the original system from \dots, however software implementations of MNA often work with $\mb{G}$ as defined in \dots\ (e.g.~when using Newton's method for solving nonlinear systems).}'' (p.2)
        \end{itemize}
    \end{itemize}

    We are confident that, considering our changes and additions, the manuscript now meets all acceptance criteria.
\end{document}
